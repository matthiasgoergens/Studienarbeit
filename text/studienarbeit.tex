\documentclass[a4paper]{amsart}

%\usepackage{palatino}
%\usepackage{euler}
%\usepackage{charter}
%\usepackage{computermodern}

\usepackage{amsfonts}

\usepackage[utf8]{inputenc}
\usepackage[T1]{fontenc}

\usepackage[english]{babel}
\usepackage{amsmath}
\usepackage{amsthm}
\usepackage{amssymb}
\usepackage{amsfonts}
% what was it for?
%\usepackage{bbm}

\usepackage{listing}

\newtheorem{theorem}{Theorem}
\theoremstyle{lemma}
\newtheorem{lemma}{Lemma}
\theoremstyle{definition}
\newtheorem{defn}[theorem]{Definition}
\theoremstyle{remark}
\newtheorem{remark}[theorem]{Remark}

\usepackage{units}
%\usepackage{icomma}

\usepackage{psfrag, graphicx}
\usepackage{url}
\usepackage{multirow}
\usepackage{multicol}
\usepackage{longtable}
\usepackage{booktabs}
\setcounter{LTchunksize}{1000}



\graphicspath{{img/}}
\author{Matthias Görgens}

% ---- LATIN ----
\def\etal{\emph{et~al.}}

\def\ie{\emph{i.e.}}
\def\Ie{\emph{I.e.}}

\def\eg{\emph{e.g.}}
\def\Eg{\emph{E.g.}}

\def\vitae{vit\ae{}}
\def\apriori{\emph{a~priori}}
\def\aposteriori{\emph{a~posteriori}}

% ---- FRENCH ----
\def\naive{na\"{\i}ve}
\def\Naive{Na\"{\i}ve}
\def\naively{na\"{\i}vely}	% Okay, I know, this isn't French.
\def\Naively{Na\"{\i}vely}

\DeclareMathOperator{\In}{in}
\DeclareMathOperator{\Out}{out}
\DeclareMathOperator{\conv}{conv}


%für ein/aus-gehende kanten
\newcommand{\ina}{\ensuremath{\vec{\In}}}
\newcommand{\outa}{\ensuremath{\vec{\Out}}}

\newcommand{\lex}{\ensuremath{\prec}} % \leq_{\mathop{lex}}}}
\newcommand{\nlex}{\ensuremath{\nprec}} % \leq_{\mathop{lex}}}}

%für knoten ein/aus-gehender kanten.
\newcommand{\inv}{\ensuremath{\dot{\In}}}
\newcommand{\outv}{\ensuremath{\dot{\Out}}}

\newcommand\mpar[1]{\marginpar {\flushleft\sffamily\small #1}}
\setlength{\marginparwidth}{3cm}
\setlength{\marginparpush}{1cm}

\newcommand{\ol}[1]{\overline{#1}}
\newcommand{\todo}[1]{\mpar{#1}}
%\newcommand{\todo}[1]{\footnote{#1}}

% lr stands for left,right
\newcommand{\lr}[1]{\ensuremath{\left( #1 \right)}}
% and for equivalency classes
\newcommand{\lrE}[1]{\ensuremath{\left[ #1 \right]}}
% and Mengen
\newcommand{\lrM}[1]{\ensuremath{\left\{ #1 \right\}}}
\newcommand{\lrX}[1]{\ensuremath{\left< #1 \right>}}

\newcommand{\abs}[1]{\ensuremath{\left| #1 \right|}}

\newcommand{\reals}{\ensuremath{\mathbb{R}}}
\newcommand{\naturals}{\ensuremath{\mathbb{N}}}
\newcommand{\rationals}{\ensuremath{\mathbb{Q}}}

\newcommand{\Hom}{\ensuremath{\mathrm{Hom}}}

\newcommand{\M}{\ensuremath{\mathcal{M}}}
\newcommand{\calO}{\ensuremath{\mathcal{O}}}
\newcommand{\calS}{\ensuremath{\mathcal{S}}}
\newcommand{\Sym}{\ensuremath{\mathfrak{S}}}

\newcommand{\SymRC}{\ensuremath{\Sym_R \times \Sym_C}}

\usepackage{hyperref}
\begin{document}

\title{Partinioning Orbitopes for Permutation of Rows and Columns -- Term Paper}
\maketitle
%\section{Abstract}

\section{Introduction}
Symmetry pops up in places as diverse as aesthetics, and conserved
physical quantities.  Or algorithmic problem solving: Say, you are
given a problem that consists of many independent but similar
subproblems, like doing some calculation with all numbers in a list.
Your solution can exploit the problem's symmetry of repetition: ``Just
do the same operation for each element of the list and collect the
results.''% -- or, if you want to be more formal and use a programming
%language (\eg{} Haskell): \todo{Oder lieber mit Funktoren?  Oder ganz informell?}
%%TeXify Haskell
%\begin{verbatim}
%solution :: [Input] -> [Output]
%solution = fmap operation
%\end{verbatim}
%You may even calculate in parallel.

But symmetry can also be a problem.  Take Buridan's ass.\todo{Weniger
  Schmalz.}  \todo{Recherche, about name and so on} Placed exactly at
the midpoint between two equally enticing haystacks, the philosopher's
ass can not decide where to go and starves.  Of course even if such an
exact placement was possibly, no real donkey would
have a problem making an arbitrary choice for one of the stacks.
However contrived the relation to a real animal, Buridan's ass does
bear some resemblance to computer programs.  One has to teach them
explicitly how to make arbitrary decisions to get out of these traps.

Programs for solving linear integer optimization problems have often
been caught between haystacks, forced to evaluate all possibilities
and not deciding until the last moment.  The introduction and analysis
of orbitopes has been one recent step towards fashioning those
programs with more judgement to cut short their search for the
optimum.  Orbitopes exploit the situation where one can introduce
groups acting on the feasible solutions of an integer problem such
that members of the same orbit share the same objective value.
Obviously, any optimization algorithm that only considers at least one
representative of each orbit will still find an optimal solution, if
there is one.  Orbitopes are the convex hulls of all representatives
thus considered.  It is customary to take the lexicographically
largest member of each orbit as its representative.

I will consider a certain kind of partitioning orbitopes.
Partitioning orbitopes are polytopes that live in the set of
\(0/1\)-matrices with exactly, one 1-entry per row.  Matrices that can
be transformed into each other by permutation of columns are
considered equal and belong to one orbit.  The partitioning orbitopes
are inclusion minimal polytopes that contain one and only one member
from each orbit: the unique representative matrix with
lexicographically descending sorted columns.  (Henceforth we will
silently imply lexicographically whenever we speak of ``biggest'', or
``least'', and furthermore imply descending order with ``sorting'', .)

In \cite{faenza-2008} Faenza and Kaibel give compact extended
formulations for the packing and partitioning orbitopes.
I want to review symmetric partitioning orbitopes: In addition
to column permutation allowed for mere partitoning orbitopes, row
permutations are allowed as well.  I will give an extended formulation
for symmetric partitioning orbitopes with full row symmetry. And I
will show that optimization over symmetric partitioning orbitopes can
be an NP-hard problem with certain other groups acting on the rows.


% Consider the very special instance of the partioning problem of
% partitioning the $p$ nodes of complete graph into at most $q$ parts.
% One well-known formulation uses variables $x_{ij}$ to indicate whether
% node $i$ is put into part $j$.

\section{Definitions}
We will use the notation introduced in \cite{orbi} by Pfetsch
and Kaibel.  Let \(\mathbbm{1}_{p,q} \in \lrM{1}^{\lrE{p}
  \times \lrE{q}}\) be the matrix composed of one-entries of size \(p \times q\).
Let \(\M_{p,q} = \{0, 1\}^{p \times q} \). 

\begin{defn}[Partitioning and Packing Matrices]
  For some size \(p, q \in \naturals^{>0}\) consider the sets of
  matrices \(\M_{p,q}^{=},\M_{p,q}^{\leq} \subseteq \M_{p, q}\), such
  that 
  \begin{align}
    \M_{p,q}^{=} & := \lrM{m \in \M_{p,q} \mid m \cdot \mathbbm{1}_{q, 1} = \mathbbm{1}_{p, 1}} & \text{(Partitioning Matrices)}\\
    \M_{p,q}^{\leq} & := \lrM{m \in \M_{p,q} \mid m \cdot \mathbbm{1}_{q, 1} \leq \mathbbm{1}_{p, 1}} & \text{(Packing Matrices)}
  \end{align}
  (\Ie{} consider the matrices with exactly one 1-entry per row and the matrices with at most one 1-entry per row.)
\end{defn}

 \begin{defn}[Orbitopes]
   Let \(G\) be a group acting on a set \(A \subseteq \rationals^{p,q}\).  Let \(A\) have a total order \(\lex\), such that \(a\lex b \lex a\) implies \(a = b\).
   Look at the set \(V \subseteq A\) uniquely determined by
   \begin{align}
     \label{biggest}
       G \cdot v  \preceq v  \text{ for all } v \in V
     \end{align}
   and
   \begin{align}
     \label{complete}
     G \cdot V =  A \text{.}
   \end{align}
   Call its convex hull \(O_{\lex} \lr{G, A} := \conv{}  V\) the orbitope of
   \(G\) on \(A\) with ordering \(\lex\).  
 \end{defn}
\begin{remark}
  We will omit stating the ordering and imply lexicographic ordering
  on the matrices, where entries with lexicographically lower indices
  are considered more significant.  For example
  \[\lr{\begin{matrix}
      1 & \mathbf{2} \\
      3 & 4 \\
      \end{matrix}
    }
    \lex
    \lr{\begin{matrix}
      1 & \mathbf{3} \\
      2 & 4 \\
      \end{matrix}
    }\text{.}
  \]
\end{remark}
\begin{remark}
  Our definition of orbitopes is slightly more generalized than the definition in \cite{orbi}.
\end{remark}

Kaibel and Pfetsch talk about orbitopes for partitioning and packing
matrices.  In associated problems the labels of the partitions can
often be exchanged without altering the quality of a solution.  For
example in the graph vertice coloring problem the colors can be
permuted.  When one describes such a coloring as a binary matrix,
where each column relates to a color and each row relates to a vertice
of the graph; then one can let a symmetric group act on the columns
(of course).

\begin{defn}[Columns and Rows]
  Fix \(p, q \in \naturals^+\).  Consider the set of matrices \(M :=
  \mathbb{C}^{p \times q}\).  Let \(R_i := M_{i, *}\) be the \(i\)-th
  row of those matrices, and let \(R\) be the set of all rows.
  \(\Sym_R\) shall designate the symmetric group over them.  Define
  \(C_j\), \(C\) and \(\Sym_C\) in analogue for the columns.  As an abuse of
  notation --- if it is clear from the context that a specific matrix
  \(m \in M\) is meant, \(R\) and \(C\) can also refer to the rows
  and columns of that specific matrix.  We will also usually supply \(p\)
  and \(q\) by context only.
\end{defn}

% Often we face a optimization problems over partitioning
% matrices where the action of \(\Sym_C\) on the columns conserves
% feasibility and objective values; then restricting the space of
% solutions to the orbitope of \(\Sym_C\) on \(O^=\) conserves the
% occurrence\footnote{and non-occurrence} of all objective values ---
% obviously including the optimum.  Thus the following definition:

%More formally the group \(S = S_R \times S_C\) acts on the solution,
%where \(S_R\) and \(S_C\) are the symmetric groups on rows and
%columns.  The next section gives a description of the representatives
%contained in a symmetric partitioning orbitope.

\begin{defn}[Packing and Partitioning Orbitopes]
  We will work with the following two orbitopes
  \begin{description}
    \item[Partitioning orbitope] \(O^=_C := O \lr{S_C, \M_{p,q}^=}\)
    \item[Packing orbitope] \(O^{\leq}_C := O \lr{S_C, \M_{p,q}^=}\)
   \end{description}
\end{defn}

We will have a look at even more symmetric problems.  Here not only
the columns can be permuted but also the rows.  For example take
vertex coloring on a graph whose automorphism group is isomorphic to
the symmetric group on its vertices.  Rows correspond to vertices and
columns to colors.

\begin{defn}[Symmetric Partitioning and Packing Orbitopes]
%  \label{espo}
Introduce the following orbitopes:
\begin{description}
\item[Symmetric partitioning orbitope] \(\calS^=_{p,q} := O^=_{p, q} \lr{\SymRC} \)
\item[Symmetric packing orbitope] \(\calS{S}^\leq_{p,q} := O^\leq_{p, q} \lr{\SymRC} \)
\end{description}
\end{defn}



%\begin{defn}[Partial Symmetric Partitioning Orbitope]
%  Add definition
%\todo{add definition}
%\end{defn}

\section{Vertices of the Symmetric Partitioning Orbitope}
Let's take a closer look at the vertices of the symmetric partitioning
orbitope \(\calS^=_{p, q}\) of size \(p, q \in \naturals^{>0}\)V.

\begin{remark}
  Note that \(\calS^=_{p, q} := O^=_{p, q} \lr{\SymRC} \subsetneq
  O^=\lr{\Sym_C,\M^=_{p, q}} \cap O^=\lr{\Sym_R, \M^=_{p, q}}\).  In
  general expressing the orbitope of an internal direct product in
  terms of the orbitopes of both factor groups is a hard problem.
\end{remark}

\begin{lemma}[Characterizing vertices of \(\calS^=_{p, q}\)]
  \label{charVert}
  Look at \(v \in \M^=_{p, q}\).  \(v\) is a vertice of \(\calS^=_{p,
    q}\), iff the following conditions hold:
  \begin{enumerate}
  \item \label{cond1} The number of zeros left of the 1-entry in
    successive rows increases by zero or one:
    \begin{equation*}
      v_{i,j} = 1\) \implies 1 \in \lrM{v_{i+1,j}, v_{i+1,j+1}} \text{
        for all }\lr{i,j} \in \lrE{p-1} \times \lrE{q} \text{.}
    \end{equation*}
    (Where the outside entries \(v_{*,q+1}\) are presumed zero to
    avoid dealing with edge cases).
  \item \label{cond2} And the number of 1-entries per column decreases
    from left to right:
    \begin{equation*} \mathop{sum} \lr{v_{*, j}} \geq \mathop{sum}
      \lr{v_{*, j+1}}\) \text{ for all \(j \in \lrE{q-1}\).}
    \end{equation*}
  \end{enumerate}
\end{lemma}

\begin{proof}
  We will prove lemma \ref{charVert} in two parts.  First we show that
  \(\calS{}^=_{p, q}\) has no vertices other than described.  Then we
  show that all the elements of \(\M^=_{p, q}\) that meet the
  conditions are truly vertices of \(\calS^=_{p, q}\).  \todo{Alternativ: Zeige
    dass die vorgeblichen Vertices \lex unter allen \(\M^=_{p, q}\)
    minimieren.}

  Take \(v \in \M^=_{p, q}\).  Suppose \(v\) violates condition
  \ref{cond1}.  Then we can find another matrix \(w\) in the orbit of
  \(v\) under \(\SymRC\), that is lexicographically bigger than \(v\):
  \(v \prec w := g \cdot v\), where \(g \in \Sym_{R \times C}\).  Thus
  \(v\) does violates \ref{biggest} in the definition of an
  orbitope.  % \(v \nin \calS^=_{p, q}\).
  But how do we find \(w = g \cdot v\)?  Consider the topmost
  violation, say \(v_{i,j} = 1\) and \(v_{i+1,k} = 1\) with \(k \notin
  \lrM{j,j+1}\) and \(i \in \lrE{p-1}\) minimal.  Look at two
  cases. In the first case the topmost violation sends the 1-entry to
  the left: \(k < j\).  Here we swap the rows \(v_{i,*}\) and
  \(v_{i+1,*}\) to obtain a lexicographically bigger matrix in the
  same orbit.  In the second case the topmost violation sends the
  1-entry too far to the right: \(k > j+1\).  Here we swap the columns
  \(v_{*,k}\) and \(v_{*,j+1}\) to obtain a lexicographically bigger
  matrix.  This matrix is really lexicographically bigger, since we
  are dealing with the topmost violation --- so there can be no
  1-entry in those columns above the \(i\)-th row.

  Now suppose \(v \in \M^=_{p, q}\) satisfies condition \ref{cond1}, but violates
  condition \ref{cond2}.  Look at the leftmost violation:
  \(s := \mathop{sum} \lr{v_{*,j}} < \mathop{sum} \lr{v_{*,j+1}} := S\) where \(j \in
  \lrE{q}\) minimal.  Let all 1-entries of those columns lie in
  \(v_{\lrE{a,a+s-1},j}\) and \(v_{\lrE{a+s,a+s+S-1},j+1}\).  By
  swapping \(v_{*,j}\) and \(v_{*,j+1}\) with a function called \(h\) and rotating rows via the permutation:
    \[\lr{f \lr{v}}_{*,i} :=  v_{*,\lrX{a, a+1, \dots, a+s+S-1}^S \lr{i}}
  \]
  we can make those 1-entries lie in the ranges of indices
  \({\lrE{a,a+S-1} \times \lrM{j}}\) and \({\lrE{a+S,a+s+S-1} \times
    \lrM{j+1}}\) and keep everything else constant.  The new matrix
  \(\lr{f \circ h} \lr{v}\) is lexicographically bigger than the old
  one \(v\).

  \todo{Introduce ranges of indices as first-class values?}
  
  We have seen that all candidates for vertices of the orbitope
  \(\calS^=_{p, q}\) have to satisfy conditions \ref{cond1} and
  \ref{cond2}.  Do these conditions suffice?

  Yes, they do.  Because the operations in \(\SymRC\) preserve exactly
  the distribution of column-sums of a matrix.  I.e. two matrices in
  \(\M^=_{p, q}\) share the same distribution of column-sums, iff they
  share the same orbit under \(\SymRC\).  And since the conditions of
  the lemma describe a straight-forward normal form, one can easily
  construct for each orbit the unique member satisfying the conditions.
\end{proof}

Computer generated examples suggest that no polynomial in \(p \cdot{}
q\) limits the number of facets of the symmetric partioning orbitope.

But - I will show an extended formulation for \(S^=_{p, q}\) by linear
inequalities in a higher dimension that is polynomially bounded in
number of both variables and facets in \(p \cdot q\).

% \section{Facets of the Simple Symmetric Partitioning Orbitope}
% %Bild!

% The simple symmetric partitioning orbitope permits an efficient
% description by equalities and inequalities in the original space of
% variables.

\section{Polynomial Optimization on the  Symmetric Partitioning
  Orbitope}
\label{fluss}
We will reduce maximization over the orbitope to finding a longest
\(s\)-\(t\)-path in an acyclic weighted digraph.

\todo{Bild f\"ur die Konstruktion des Graphen!}

\subsection{Setup}

Let \(p, q \in \mathbb{N} \setminus \{0\}\).  Given a matrix \(M \in
\mathbb{Q}^{p \times q}\) of objective values, consider the program:
\begin{equation}
\label{optS}
\max \lrX{M, \mathbf{x} }  \text{ s.t. } x \in S^=_{p, q}
\end{equation}
where \[\lrX{M, \mathbf{x} } := \sum_{(i, j)\in [p]\times [q]} x_{i, j}\cdot M_{i, j}\text{.}\]
We will construct an acyclic weighted graph \(D := (V, A)\) to project each vertex of
\(S^=_{p, q}\) and its objective value to one \(s\)-\(t\)-path in \(D\) and its length.

Define the set of nodes:
\[V := \lr{ [p]\times [q] \times [p]} \uplus \{s\} \uplus \{t\}\text{.}\] 
Each \textit{ordinary} node, that is
each node besides \(s\) and \(t\), encodes a row, a column and the
maximum Hamming weight per column still allowed.

The arcs \(A := \overline{A} \uplus A_t\) can be described as a union
of ordinary arcs \(\overline{A}\) and arcs \(A_t\) ending in \(t\).  Setting
\(s:=(0, 0, q)\) allows the arcs from $s$ to be formalized as
ordinary arcs.
\begin{align}
  \overline{A} &:=
  \lrM{ \lr{r, c, h} \rightarrow \lr{r+h', c+1, h'} \in V \times V \colon
  h' \leq h }} \\
  A_t &:= ([p] \times {q} \times [p]) \times \{t\}
\end{align}

For parent and child nodes and arcs we will use the following notation:
\begin{align*}
\ina\colon  V &\to 2^A \\
v &\mapsto \lr{V \times \{v\}} \cap A\\
%\end{align*}
%\begin{align*}
\outa\colon  V &\to 2^A \\
v &\mapsto \lr{\{v\} \times V} \cap A\\
%\end{align*}
%\begin{align*}
\inv\colon  V &\to 2^A \\
v &\mapsto \lrM{ u \colon \lr{u, v} \in \ina(v) }}\\
%\end{align*}
%\begin{align*}
\outv\colon  V &\to 2^A \\
v &\mapsto \lrM{w \colon \lr{v, w} \in \outa\lr{v} }\\
\end{align*}
(Where \(2^A\) means the set of all subsets of \(A\).)

To link objective values over \(S^=_{p, q}\) with
\(s\)-\(t\)-path-lengths in \(D\) we introduce arc weights \(m\colon A \to \mathbb{Q}\).
\begin{align}
\label{objLink}
  m\lr{\lr{r, c, h} \rightarrow \lr{r+h', c+1, h'} } & := \sum_{i=r+1}^{r+h'} M_{i, c+1} \\
  m\lr{\lr{r, c, h} \rightarrow t } & := 0
\end{align}

\subsection{Finding the Longest \(s\)-\(t\)-Path in \(D\)}

Call \(P_s \colon V \to A^* \) the function that maps each node \(v\)
to the longest path connecting \(s\) and \(v\).  Here \(A^*\) means
the Kleene closure of \(A\): I.e. the set of all strings of arcs \(a
\in A\) including the empty string.  Also let
\begin{align*}
m \colon A^* &\to \mathbb{R} \\
\omega &\mapsto \sum_{a\in \omega} m \lr{a }
\end{align*}

To find the longest \(s\)-\(t\)-path in \(D\) is to calculate
\(P_s (t)\).  Dynamic programming lends itself to this task.  Since
\(D\) is acyclic we start knowing \(P_s (s) = \lambda\).

For some nodes \(v \in V\) the paths \(P_s\) will be known for all its parents \(\inv{} \lr{v} \).  We conclude:
\[m\lr{P_s \lr{v}} = \max_{w \in \inv(v)}
m\lr{P_s(w) | \lr{w \rightarrow v }}\text{.}\]
The graph \(D\) admits at least one trivial path from \(s\) to \(t\).
Together with the acyclic and finite nature of \(D\) this guarantees that we
learn \(P_s (t)\) eventually.  More specifically we get to know \(P_s
(t)\) after looking at each arc of \(D\) at most a constant number of
times, using dynamic programming.

%the
%following dynamic programming technique: Since \(D\) is an acyclic
%digraph we can sort its vertices topologically.

%Afterwards for each node \(v\) in topological order we calculate the
%longest \(s\)-\(v\) path and distance based on the known longest
%\(s\)-\(w\)-distances

%Topologische Sortierung, 

\todo{Bilder!}

\section{Extended Formulation for the Symmetric Partitioning Orbitope}
The aforementioned Graph \(D\) can be used to derive an extended
formulation for the  symmetric partitioning orbitope.  For this
formulation we have to construct a linear description of finding the
longest path in a digraph.  Fortunately -- as it is well known --
network flows do the trick.  The transformation from network flows
back to the original variable space of \(S^=_{p, q}\) can be done with
a linear function, as well.

We introduce flow variables \(f \in \mathbb{R}^A_{\geq 0}\).  

For sets \(M\) we define
\[f (M) := \sum_{m\in M} f(m)\text{.}\]

Now, we can begin with the extendend formulation for the 
symmetric partitioning orbitope.  For each node \(v \in V \setminus
\{s, t\}\) we have:
\begin{equation}
f\lr{\ina(v)} = f\lr{\outa(v)}
\end{equation}
and for \(s\) we get:
\begin{equation}
f\lr{\outa(s)} = 1
\end{equation}

% subto coupling:
% forall <sx, cx> in Y:
%        y[sx,cx] == sum <s, c, h, s_, c_, h_> in A
%               	   with c+1 == cx
% 		   and cx == c_
% 	    	   and s < sx and sx <= s_ :
% 	   	   f[s, c, h, s_, c_, h_];


To link \(f\) to the \(x\) of our original problem (\ref{optS}) we
introduce the following conditions that are in analogue to \ref{objLink}:
%\todo{nachschauen, wie man die Indexbereiche unter der Summe stapelt}
\begin{equation}
  x \lr{ r, c }=
  \sum_{r' - h' \leq r \leq r'; h' \leq h} f\lr{\lr{r'-h', c-1, h}
    \rightarrow \lr{r', c, h'} }
\end{equation}

\section{Restricting the Symmetric Group}

Allowing permutations of all columns and a subset of rows produces an
NP-hard optimization problem in general.  This can be shown by
reduction from another problem, that is already proven to be NP-hard.

\begin{defn}[Partial symmetric partitioning orbitope]
  \label{def-subset-rows}
  Let \(p, q \in \naturals^+\).  Fix a subset \(J \subseteq \lrE{p}\).
  Call \(\Sym_{J}\) the symmetric group acting on \(R_J\) of
  \(\M^=_{p, q}\).  Define the partial symmetric partitioning orbitope
  as \(\mathcal{S}^=_{p,q}\lr{J} := O^=_{p, q} \lr{\Sym_{J} \times
    \Sym_{C}} \).
\end{defn}

Vertices of \(\calS^=_{p,q}\lr{J}\) are lexicographically maximal elements of orbits of \(\M^=_{p,q}\) under \(\Sym_{J} \times
    \Sym_{C}\).  Their upper \(J\) rows satisfy the conditions of lemma
\ref{charVert}, \ie \(\lr{\calS^=_{p,q} \lr{J} }_{*,J} = \calS^=_{p,J}\).  This is because all elements of the lower
rows \(l \subseteq \M^=_{p,  \lrE{q}\setminus J}\) have lexicographically less significant
indices than the upper part.

Below the structure is more complicated: Consider the set \(l\lr{u} :=\lrM{ m  \mid  \lr{\begin{matrix}u\\m\end{matrix}} \in \calS^=_{p,q}\lr{J}}\)
of lower rows that complement a given upper part \(u\in\calS^=_{J,q}\).
The columns are partitioned naturally by the sum of their entries in the upper part \(u\): \(E_s = \lrM{\lr{i,*} \mid \mathop{sum} \lr{u_{i,*}} = s}\).

\todo{Weitermachen!  \(\sym{E_s} \dots\)}

%For a set matrix \(m \subset \(\calS^=_{p,q}\lr{J}\)\) fixed upper part let
%\(E\) be a partition of rows by the sum of entries in their upper \(J\) rows


\todo{Describe lexicographically maximal elements of the orbits of
  \(\Sym_J \times \Sym_C\)}

Let's examine the vertices of
\(\mathcal{S}^=_{p,q}\lr{J}\), \ie{} the lexicographically maximal
elements of the orbits of \(\Sym_J \times \Sym_C\) acting on
\(\M^=_{p,q}\).

\begin{theorem}[Subset of rows]
  \label{subset-rows}
  Optimizing a linear function \(d \in \Hom\lr{\rationals^{p,q},\rationals}\) over the partial symmetric partitioning orbitope
  \(\mathcal{S}^=_{p, q}\lr{I}\) is NP-hard.
\end{theorem}

For the proof of theorem \ref{subset-rows}, we will need a result by
Yuri Faenza\todo{Wer noch? + Quelle.  Nachfragen!}:

\begin{theorem}[``Pairwise permutations of columns are NP-hard'']
  \label{pairwise}
  For some \(p, q \in \naturals^+\) consider the partitioning matrices
  \(\M^=_{p \times q}\).  \todo{Ist das so korrekt formuliert?}  Let
  \[P_I := \prod_{i=1}^{\lceil q/2 \rceil}
\Sym_{\lrM{C_{2i-1}, C_{2i}}}\]
  be the cartesian product of symmetric groups whose action swaps pairs of adjacent columns.
  Optimizing a linear objective function \(c \in \Hom\lr{\rationals^{p \times
    q}, \rationals}\) over \(O^=_{p, q} \lr{P_I}\) is NP-hard.
\end{theorem}

\begin{proof}[Proof of `Subset of rows']
  We prove theorem \ref{subset-rows} by reducing the problem given in
  theorem \ref{pairwise} to the sub\/set-\/of-\/rows problem: Given the linear
  objective function in \(\Hom \lr{\M^=_{p, q}, \rationals}\), we
  will extend it upwards to a function in \(\Hom \lr{\M^=_{e+p, q},
    \rationals}\) on a set of much taller partitioning matrices.  The
  extra co\"efficients at the top will encode the structure of the
  disjoint pairs of columns: We will force both elements of any
  commuting column pair in the original problem, to have the same
  number of ones in the upper part of the extended matrix.
  %% Since we only include lexicographically maximal elements of 

  Fix a size \(p, q\), an objective function \(c\) and pairs of
  columns \(I\) as in the statement of theorem \ref{pairwise}.  Assume
  -- without loss of generality -- that we are maximizing \(c\) on
  \(\M^=_{p,q}\).  Without fixing \(e\) yet, embed \(\M^=_{p,q}\) at
  `the bottom of' \(\M^=_{e+p,q}\), \ie
  \begin{align}
    \varphi \colon \M^=_{p,q} & \to \M^=_{e+p,q} \\
    \nonumber
     m & \mapsto \lr{\begin{matrix}
       \mathbf{0} \\
       m
     \end{matrix}}
 \end{align}
 And to get back, use the projection \(\pi\):
 \begin{align}
   \pi \colon \M^=_{e+p,q} & \to \M^=_{p,q} \\
    \nonumber
     \lr{\begin{matrix}
         l \\
         m
     \end{matrix}} & \mapsto m
  \end{align}
  We will call the complementary projection \(\pi^\bot\):
  \begin{align}
    \pi^\bot \colon \M^=_{e+p,q} & \to \M^=_{e,q} \\
    \nonumber
    \lr{\begin{matrix}
        l \\
        m
      \end{matrix}} & \mapsto l
  \end{align}
  Let \(J := \lrE{e}\) and define \(\Sym_{J}\) as in
  \ref{def-subset-rows}.  How can the upper \(e\) rows encode \(I\)?
  
  We want to force the upper \(e\) rows so that columns that permute
  in \(P_I\) have the same number of 1-entries.  The number of
  1-entries per column can not increase as we go from left to
  right. \todo{Describe the vertices above.}

  So we choose the number \(\Delta\lr{j}\) of ones per column \(j\) such that
  \begin{align}
    \Delta \lr{j+1} & =  \Delta \lr{j} - \begin{cases} 0 & \text{if } j \in I\\
      1 & \text{otherwise} \\
    \end{cases}\\
    \Delta \lr{q} & = 0 \text{.}
  \end{align}
  Now, number of zeros \(o\lr{j}\) in a column \(j\) above the first one is governed by the following equation
\begin{align}
  o \lr{j} & = \sum_{k \in \lrE{j-1}} \Delta \lr{k} \\
 \end{align}
 Choose \(e := o\lr{q} + \Delta\lr{q} \leq \frac{q \lr{q+1}}{2}\) as
 the smallest number of rows that can accommodate all ones as given
 above.
 
 Now only the objective function \(d\) is left to be defined, so that it
 actually enforce the structure of ones and zeroes that we desire:
 \begin{align}
   d & \in \Hom \lr{\rationals^{\lr{e + p} \times q}, \rationals} \\
   \nonumber
   d_{i,j} & := \begin{cases} s+1 & \text{if } o\lr{j} < i \leq o\lr{j}+\Delta\lr{j}\\
     c_{i-o\lr{q},j} & \text{if } i \geq o\lr{q}\\
     -s-1 & otherwise
     \end{cases}
 \end{align}
 Where \(s\) has to be chosen large enough, to force the upper part of
 the partition matrix no matter what the rest does:
 \begin{align}
   s := p \cdot \lr{\max_{i,j \in \lrE{p}\times \lrE{q}} c_{i,j}
     - \min_{i,j \in \lrE{p}\times \lrE{q}} c_{i,j}}
 \end{align}
 \(\lrX{d}+\lrX{J}\) is certainly polynomial in \(\lrX{c} + \lrX{I}\).  And to get from
 \begin{align}
   \hat{d} := \max d \lr{m} \text{ s.t. } m \in \mathcal{S}^=_{p, q}\lr{I}
 \end{align}
 to
 \begin{align}
   \hat{c} := \max c \lr{m} \text{ s.t. } m \in O^=_{p, q} \lr{P_I}
 \end{align}
 \todo{Why does the lower part of the new matrix resemble the old matrix?} 
 you just exploit
 \begin{align}
   \hat{c} = \hat{d} - s \cdot e \text{.}
 \end{align}

\end{proof}

% For a natural number \(a \leq q\), let \(G_{Rows} = S_a\) act on
% the first \(a\) rows.  As usually \(G_{Columns} = S_p\) acts on
% all columns.  Look at the orbits of \(G_{Columns} \times G_{Rows}\)
% and take each orbit's lexicographically smallest member as its
% representative.

% \[\lrE{\lr{\begin{matrix}
% 1 & 0 & 0 \\
% 0 & 1 & 0 \\
% 0 & 1 & 0 \\
% 0 & 0 & 1\\
% \end{matrix}
% }}
% \]


% In a representative the columns will be sorted, and the last rows,
% too.  But we can demand even more.  Say \(a = 3\), then the following
% two matrices satisfy both condition, but are still in the same
% equivalency class:

% \[\lrE{\lr{\begin{matrix}
% 1 & 0 & 0 \\
% 0 & 1 & 0 \\
% 0 & 1 & 0 \\
% 0 & 0 & 1 \\
% \end{matrix}}} = \lrE{\lr{\begin{matrix}
% 1 & 0 & 0 \\
% 0 & 1 & 0 \\
% 0 & 1 & 0 \\
% 0 & 0 & 1 \\
% \end{matrix}}}
% \]



% [1 0 0
% ,0 1 0
% ,0 1 0
% ,0 0 1]

% [1 0 0
% ,0 1 0
% ,0 0 1
% ,0 0 1]

% S_t auf t-elementiger Teilmengen der Zeilen:
% z.B. erste t-Zeilen,
%      letzte t-Zeilen
% beliebige t-Zeilen (erstmal t=2)

%\section{Cartesian Products of Symmetric Groups}
%Kartesisches Produkt von symmetrischen Gruppen (vereinfachung?)

\bibliographystyle{plain}
\bibliography{quellen}

\end{document}
