\documentclass{article}

\usepackage[ngerman]{babel}
\usepackage{a4}
\usepackage[utf8]{inputenc}
\usepackage[T1]{fontenc}
\usepackage{amsmath}
\usepackage{amsfonts}
\usepackage{amssymb}

\usepackage{units}
\usepackage{icomma}

\usepackage{psfrag, graphicx}
\usepackage{url}
\usepackage{multirow}
\usepackage{multicol}
\usepackage{longtable}
\usepackage{booktabs}
\setcounter{LTchunksize}{1000}


\graphicspath{{img/}}
\author{Matthias Görgens}

\begin{document}

\title{Partinioning Orbitopes for Permutation of Rows and Columns -- Term Paper}
\maketitle
\section{Abstract}

\section{Introduction}
Symmetry is a problem in integer programming.  The introduction and
analysis of orbitopes has been one recent step towards removing those
symmetries.  Orbitopes exploit the fact that one can introduce groups
acting on the feasible solutions of a symmetric integer problem such
that members of the same orbit share the same objective value.
Obviously, any optimization algorithm that only considers at least one
representative of each orbit will still find an optimal solution, if
there is one.  Orbitopes are the convex hulls of all representatives
thus considered.

In this paper I will consider a certain kind of partitioning
orbitopes.  Partitioning orbitopes are polytopes that live in the
set of $0/1$-matrices with at most, resp. exactly, one 1-entry per
row.  Matrices that can be transformed into each other by permutation
of columns are considered equal and belong to one orbit.  The
partitioning orbitopes are inclusion minimal polytopes that contain
one and only one member from each orbit: the unique representative
matrix with lexicographically sorted columns.

In \cite{faenza-2008} Faenza and Kaibel give compact extended
formulations for the packing and partitioning orbitopes.  In this
paper I want to give extended formulations for symmetric partitioning
orbitopes: In addition to column permutation, row permutations are
allowed as well.  More formally the group \(S = S_R \times S_C\) acts
on the solution, where \(S_R\) and \(S_C\) are the symmetric groups on
rows and columns.  The next section gives a description of the
representatives contained in a symmetric partitioning orbitope.

% Consider the very special instance of the partioning problem of
% partitioning the $p$ nodes of complete graph into at most $q$ parts.
% One well-known formulation uses variables $x_{ij}$ to indicate whether
% node $i$ is put into part $j$.

\section{Vertices of the Symmetric Partitioning Orbitope}
Let $p$ be the number of columns and $q$ be the number of rows of
matrices in the set \(M \subseteq \left\{0,1\right\}^{p\times q}\) of all
$0/1$-matrices with exactly one 1-entry in each row.  Let \(S = S_R
\times S_C\) be the Cartesian product of the symmetric groups on rows
and columns.  \(S\) acts on \(M\).

% Sortierung der Zeilen reicht auch schon für simple s.p.o.

Obviously we can require representatives to be lexicographically
sorted along rows and columns.  Thus we get representative matrices
with 1 in the top-left.  In each row the 1-entry stays in the column
it dwelled in the row above -- or wanders off one column right.

Using a very simple flow network one can easily find the facettes of
the resulting \textit{simple symmetric partition orbitope} -- see
\ref{fluss}.

Unfortunately this way some orbits of $S$ intersect the orbitope with
more than one member.  To guarantee exactly one representative per
orbit we can require runs of 1-entries in the same column to descend
in length.

Computer generated examples suggest that no polynomial in \(p \cdot
q\) limits the number of the resulting \textit{exact symmetric partioning
orbitope}'s facettes.

But - I will show an extended formulation for this exact symmetric
partitioning orbitope that is polynomially bounded in number of both
variables and facettes in \(p \cdot q\).

\section{Facettes of the Simple Symmetric Partitioning Orbitope}
%Bild!

The simple symmetric partitioning orbitope permits an efficient
description by equalities and inequalities in the original space of
variables.

% 1  1  1
%-1 -1 -1

%-1 -1 -1
% 1  1  1  1

\section{Polynomial Optimization on the Exact Symmetric Partitioning
  Orbitope}



\bibliographystyle{plain}
\bibliography{quellen}

\end{document}