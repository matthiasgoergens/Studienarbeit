\documentclass{amsart}

\usepackage[utf8]{inputenc}
\usepackage[T1]{fontenc}

\usepackage[english]{babel}
\usepackage{amsmath}
\usepackage{amsthm}
\usepackage{amsfonts}
\usepackage{amssymb}

\theoremstyle{definition}
\newtheorem{defn}{Definition}

\usepackage{units}
\usepackage{icomma}

\usepackage{psfrag, graphicx}
\usepackage{url}
\usepackage{multirow}
\usepackage{multicol}
\usepackage{longtable}
\usepackage{booktabs}
\setcounter{LTchunksize}{1000}

%\usepackage{palatino}
%\usepackage{euler}

\graphicspath{{img/}}
\author{Matthias Görgens}


\DeclareMathOperator{\In}{in}
\DeclareMathOperator{\Out}{out}

%für ein/aus-gehende kanten
\newcommand{\ina}{\ensuremath{\vec{\In}}}
\newcommand{\outa}{\ensuremath{\vec{\Out}}}

%für knoten ein/aus-gehender kanten.
\newcommand{\inv}{\ensuremath{\dot{\In}}}
\newcommand{\outv}{\ensuremath{\dot{\Out}}}

\usepackage{hyperref}
\begin{document}

\title{Partinioning Orbitopes for Permutation of Rows and Columns -- Term Paper}
\maketitle
%\section{Abstract}


\section{Introduction}
Symmetry is a problem in integer programming.  The introduction and
analysis of orbitopes has been one recent step towards removing those
symmetries.  Orbitopes exploit the fact that one can introduce groups
acting on the feasible solutions of a symmetric integer problem such
that members of the same orbit share the same objective value.
Obviously, any optimization algorithm that only considers at least one
representative of each orbit will still find an optimal solution, if
there is one.  Orbitopes are the convex hulls of all representatives
thus considered.

In this paper I will consider a certain kind of partitioning
orbitopes.  Partitioning orbitopes are polytopes that live in the
set of $0/1$-matrices with at most, resp. exactly, one 1-entry per
row.  Matrices that can be transformed into each other by permutation
of columns are considered equal and belong to one orbit.  The
partitioning orbitopes are inclusion minimal polytopes that contain
one and only one member from each orbit: the unique representative
matrix with lexicographically sorted columns.

In \cite{faenza-2008} Faenza and Kaibel give compact extended
formulations for the packing and partitioning orbitopes.  In this
paper I want to give extended formulations for symmetric partitioning
orbitopes: In addition to column permutation, row permutations are
allowed as well.  More formally the group \(S = S_R \times S_C\) acts
on the solution, where \(S_R\) and \(S_C\) are the symmetric groups on
rows and columns.  The next section gives a description of the
representatives contained in a symmetric partitioning orbitope.

% Consider the very special instance of the partioning problem of
% partitioning the $p$ nodes of complete graph into at most $q$ parts.
% One well-known formulation uses variables $x_{ij}$ to indicate whether
% node $i$ is put into part $j$.

\section{Vertices of the Symmetric Partitioning Orbitopes}
Let $p$ be the number of columns and $q$ be the number of rows of
matrices in the set \(M \subseteq \left\{0,1\right\}^{p\times q}\) of all
$0/1$-matrices with exactly one 1-entry in each row.  Let \(S = S_R
\times S_C\) be the Cartesian product of the symmetric groups on rows
and columns.  \(S\) acts on \(M\).

% Sortierung der Zeilen reicht auch schon für simple s.p.o.

Obviously we can require representatives to be lexicographically
sorted along rows and columns.  Thus we get representative matrices
with 1 in the top-left.  In each row the 1-entry stays in the column
it dwelled in the row above -- or wanders off one column right.  In
fact sorting along the rows is sufficent for this normal form; sorting
the columns does not add to the requirements.

Using a very simple flow network one can easily find the facettes of
the resulting \textit{simple symmetric partition orbitope} -- see
\ref{fluss}.

% sort columns by hamming distance

Unfortunately this way some orbits of $S$ intersect the orbitope with
more than one member.  To guarantee exactly one representative per
orbit we restrict the orbitope further:

\begin{defn}[Exact Symmetric Partitioning Orbitope]
  \label{espo}
  Given \(p, q \in \mathbb{N} \setminus \{0\}\), define the exact
  symmetric partitioning orbitope \(S^=_{p,q}\) as the convex hull of
  all matrices in \(\left\{0,1\right\}^{p\times q}\) whose rows are in
  lexicographically decreasing order and whose columns are in
  decreasing order by Hamming weight.
\end{defn}

Computer generated examples suggest that no polynomial in \(p \cdot
q\) limits the number of exact symmetric partioning orbitope's
facettes.

But - I will show an extended formulation for \(S^=_{p,q}\) by linear
inequalities that is polynomially bounded in number of both variables
and facettes in \(p \cdot q\).


\section{Facettes of the Simple Symmetric Partitioning Orbitope}
%Bild!

The simple symmetric partitioning orbitope permits an efficient
description by equalities and inequalities in the original space of
variables.

% 1  1  1
%-1 -1 -1

%-1 -1 -1
% 1  1  1  1

\section{Polynomial Optimization on the Exact Symmetric Partitioning
  Orbitope}
\label{fluss}
We will reduce maximization over the orbitope to finding a longest
\(s\)-\(t\)-path in an acyclic weighted digraph.

\subsection{Setup}

Let \(p, q \in \mathbb{N} \setminus \{0\}\).  Given a matrix \(M \in
\mathbb{Q}^{p \times q}\) of objective values, consider the program:
\begin{equation}
\label{optS}
\max \left<M, \mathbf{x} \right>  \text{ s.t. } x \in S^=_{p,q}
\end{equation}
where \[\left<M, \mathbf{x} \right> := \sum_{(i,j)\in [p]\times [q]} x_{i,j}\cdot M_{i,j}\text{.}\]
We will construct a acyclic weighted graph \(D := (V, A)\) to project each vertex of
\(S^=_{p,q}\) and its objective value to one \(s\)-\(t\)-path in \(D\) and its length.

Define the vertex set:
\[V := \left( [p]\times [q] \times [p]\right) \uplus \{s\} \uplus \{t\}\text{.}\] 
Each \textit{ordinary} vertex, that is
each vertex besides \(s\) and \(t\), encodes a row, a column and the
maximum Hamming weight per column still allowed.

The arcs \(A := \overline{A} \uplus A_t\) can be described as a union
of ordinary arcs \(\overline{A}\) and arcs \(A_t\) ending in \(t\).  Setting
\(s:=(0,\, 0,\,q)\) allows the arcs from $s$ to be formalized as
ordinary arcs.
\begin{align}
  \overline{A} &:=
  \left\{ \left(r,\, c,\, h\right) \rightarrow \left(r+h',\, c+1,\, h'\right) \in V \times V \colon
  h' \leq h \right\} \\
  A_t &:= ([p] \times {q} \times [p]) \times \{t\}
\end{align}

For parent and child nodes and arcs we will use the following notation:
\begin{align*}
\ina\colon  V &\to \mathcal{P}(A) \\
v &\mapsto \left(V \times \{v\}\right) \cap A\\
%\end{align*}
%\begin{align*}
\outa\colon  V &\to \mathcal{P}(A) \\
v &\mapsto \left(\{v\} \times V\right) \cap A\\
%\end{align*}
%\begin{align*}
\inv\colon  V &\to \mathcal{P}(V) \\
v &\mapsto \left\{ u \colon \left(u,v\right) \in \ina(v) \right\}\\
%\end{align*}
%\begin{align*}
\outv\colon  V &\to \mathcal{P}(V) \\
v &\mapsto \left\{w \colon \left(v,w\right) \in \outa\left(v\right) \right\}\\
\end{align*}

To link objective values over \(S^=_{p,q}\) with
\(s\)-\(t\)-path-lengths in \(D\) we introduce arc weights \(m\colon A \to \mathbb{Q}\).
\begin{align}
\label{objLink}
  m\left(\left(r,\, c,\, h\right) \rightarrow \left(r+h',\, c+1,\, h'\right) \right) & := \sum_{i=r+1}^{r+h'} M_{i,c+1} \\
  m\left(\left(r,\, c,\, h\right) \rightarrow t \right) & := 0
\end{align}

\subsection{Finding the Longest \(s\)-\(t\)-Path in \(D\)}

Call \(P_s \colon V \to A^* \) the function that maps each node \(v\)
to the longest path connecting \(s\) and \(v\).  Here \(A^*\) means
the Kleene closure of \(A\): I.e. the set of all strings of arcs \(a
\in A\) including the empty string.  Also set 
\begin{align*}
m \colon A^* &\to \mathbb{R} \\
\omega &\mapsto \sum_{a\in \omega} m \left(a \right)
\end{align*}

To find the longest \(s\)-\(t\)-path in \(D\) is to calculate
\(P_s (t)\).  Dynamic programming lends itself to this task.  Since
\(D\) is acyclic we start knowing \(P_s (s) = \lambda\).

For each \(v \in V\) for which \(P_s (v)\) is unknown but for whose
every parent \(w \in \inv(v)\) the path \(P_s (w)\) is already known,
we conclude
\[m\left(P_s \left(v\right)\right) = \max_{w \in \inv(v)}
m\left(P_s(w) \cdot \left(w \rightarrow v \right) \right)\text{.}\]

The graph \(D\) admits at least one trivial path from \(s\) to \(t\).
Together with the acyclic and finite nature of \(D\) this guarantees that we
learn \(P_s (t)\) eventually.  More specifically we get to know \(P_s
(t)\) after looking at each arc of \(D\) at most a constant number of
times.

%the
%following dynamic programming technique: Since \(D\) is an acyclic
%digraph we can sort its vertices topologically.

%Afterwards for each node \(v\) in topological order we calculate the
%longest \(s\)-\(v\) path and distance based on the known longest
%\(s\)-\(w\)-distances

%Topologische Sortierung, 

%Bilder!

\section{Extended Formulation for the Exact Symmetric Partitioning Orbitope}
The aforementioned Graph \(D\) can be used to derive an extended
formulation for the exact symmetric partitioning orbitope.  For this
formulation we have to construct a linear description of finding the
longest path in a digraph.  Fortunately -- as it is well known --
network flows do the trick.  The transformation from network flows
back to the original variable space of \(S^=_{p,q}\) can be done with
a linear function, as well.

We introduce flow variables \(f \in \mathbb{R}^A_{\geq 0}\).  

For sets \(M\) we define
\[f (M) := \sum_{m\in M} f(m)\text{.}\]

Now, we can begin with the extendend formulation for the exact
symmetric partitioning orbitope.  For each vertex \(v \in V \setminus
\{s,t\}\) we have:
\begin{equation}
f\left(\ina(v)\right) = f\left(\outa(v)\right)
\end{equation}
and for \(s\) we get:
\begin{equation}
f\left(\outa(s)\right) = 1
\end{equation}

% subto coupling:
% forall <sx,cx> in Y:
%        y[sx,cx] == sum <s, c, h, s_, c_, h_> in A
%               	   with c+1 == cx
% 		   and cx == c_
% 	    	   and s < sx and sx <= s_ :
% 	   	   f[s, c, h, s_, c_, h_];


To link \(f\) with \(x\) of our original problem (\ref{optS}) we
introduce the following conditions that are in analogue to \ref{objLink}:
%TODO: nachschauen, wie man die Indexbereiche unter der Summe stapelt
\begin{equation}
  x \left( r,\, c \right) =
  \sum_{r' - h' \leq r \leq r'; h' \leq h} f\left(\left(r'-h',\, c-1,\, h\right)
    \rightarrow \left(r',\, c,\, h'\right) \right)
\end{equation}

\section{Restricting the Symmetric Group}
In this section we will have a look at some other groups \(G =
G_{Rows} \times S_{Columns}\) operating on the solutions.  Always the
permutation of all columns will be allowed.  But we will look at
\(G_{Rows}\) being the symmetric group on the last \(t\) rows and then
on the first \(t\) rows in order.

\subsection{Permuting the last rows}


\subsection{Permuting the first rows}

% S_t auf t-elementiger Teilmengen der Zeilen:
% z.B. erste t-Zeilen,
%      letzte t-Zeilen
% beliebige t-Zeilen (erstmal t=2)

%\section{Cartesian Products of Symmetric Groups}
%Kartesisches Produkt von symmetrischen Gruppen (vereinfachung?)

\bibliographystyle{plain}
\bibliography{quellen}

\end{document}